
\abstract{The problem of identifying natural, socio-economically defined neighborhoods arises in applied contexts including Census reporting, measurement of segregation, and dimension reduction in urban computing. This problem is also of interest for urban theory, since the difficulty of identifying neighborhoods may be viewed as a measure of socio-spatial complexity. 

We develop a rigorous, information-theoretic approach to this topic, using open data on race in American cities as a case study. First, we formulate the \emph{mean local information} $J(X,Y)$ as a localization of the mutual information between spatial and socio-economic variables. The measure $J(X,Y)$ is closely related to the Fisher information of the underlying joint distribution, and is therefore a measure of the intrinsic spatial complexity of an urban phenomenon. Unlike standard global information measures, the mean local information clearly distinguishes between cities like Detroit--which is dominated by a few huge, monoracial superclusters--and cities like Philadelphia--which is an intricate patchwork of small, racially-distinct neighborhoods. Second, we provide a practical algorithm for identifying natural neighborhoods through greedy information maximization, and relate this algorithm's behavior to the mean local information. Questions raised by this work include the social, economic, and policy determinants of socio-spatial complexity in cities, and the potential use of spatial information measures in quantifying temporal changes in socio-economic structure, on time scales ranging from days to decades.}

