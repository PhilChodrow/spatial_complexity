	Information theory provides one natural approach toward thinking about complex systems. It achieves this by mapping the physical concept of structure to the epistemic concept of predictability. A complex system has enough structure to be at least partially predictable, but enough variability to make prediction challenging. Information theory's formalization of the concept of predictability is therefore tailor-made for the study of complex phenomena. 

	In this essay, we apply information-theoretic tools to the measurement of \emph{spatial compositional complexity}. A data set is \emph{spatial} when it consists of a series of measurements on a set of points $\mathcal{X}$ endowed with a measure of distance; formally, $\mathcal{X}$ is a metric space. A data set is \emph{compositional} when its observations consist in a set of counts or proportions across a set of fixed subcategories of a whole.\footnote{Our use of this term differs slightly from that used in the field of Compositional Data Analysis (CDA), in which compositional data is always proportions and never frequencies \cite{Aitchison1982,Aitchison2002}. Because our focus is on the compositional structure of these phenomena, we take the term to be appropriate in this context as well.} The alphabet of categories might be $\mathcal{Y} = \{\text{Humanities, Social Sciences, Natural Sciences}\}$, in which case an observation might consist in a tabulation of undergraduates majoring in each of these fields. 

	Our focus is on data sets that are both spatial and compositional, especially as they arise in urban theory and urban planning. Our working example is the spatial variation of racial trends in American cities. Information-theoretic concepts have already found application in urban planning problems related to zoning and predicting population distributions \cite{Royal2014,Batty1974,Batty1976,Battya}. Substantial recent work has addressed the measurement of difference and disparity in cities through an information-theoretic lens \cite{Theil1971,Bettencourt2015,Roberto2015a,Roberto2015}. Attractive features of information-theoretic measures for this purpose include their deep relationship to statistical inference \cite{Cover1991,Csiszzr2004}, their generalizability to multiple demographic phenomena, and the fact that Theil's (\cite{Theil1971}) original index satisfaction of many (though not all) of the invariance properties desirable for the measurement of segregation \cite{Reardon2002}. \nocite{Sampson2002,Dietz2002,Wong2004,Keeling1999,Anas1997,Ioannides2004a,Wong1999,Press2009a,Holloway2012,Lee2008,Louf2015,Webber1979,Bivand2014b,Bivand2014a,Bivand2014}.

	Because an operationally satisfying, mathematically unified concept of segregation remains elusive, we focus on the related concepts of spatial complexity and neighborhood structure. While these concepts may appear both harder to define and less useful in practical social science, we demonstrate in what follows that a mathematically unified approach can both give content to these concepts and shed light on some of the concerns more traditionally related to segregation studies. 

	\begin{figure}
	\centering
		\includegraphics[width=.4\textwidth]{external_figs/detroit_dot.png}
		\includegraphics[width=.4\textwidth]{external_figs/philly_dot.png}
		\caption{Dotmaps of racial composition for Detroit (top) and Philadelphia (bottom) by blockgroup according to the 2010 US Census. Racial Dotmap created by Dustin Cable and the Weldon Cooper Center at the University of Virginia.} 
		\label{fig:info_cross}
	\end{figure}	

\subsection{Structure of the Essay}
	In Section 2, we motivate and develop the two core components of our mathematical framework. The first of these is the mean local information $J(X,Y)$, a measure of the intrinsic spatial complexity of a compositional phenomenon. In the context of racial trends in cities, the mean local information measures the granularity of racial neighborhood structure. It is low in a city like Detroit, which consists of a few large, monolithic tracts of constant racial composition. It is high in a city like Philadelphia, in which many (spatially) smaller neighborhoods are knit together in an intricate patchwork. The second component of our mathematical framework is a simple algorithm for identifying clusters that are both spatially and compositionally coherent through agglomerative hierarchical clustering and greedy information maximization. In the context of race in cities, this algorithm may be viewed as an automated means to identify ``natural'' neighborhoods. 

	In Section 3, we apply these techniques to the analysis of spatial trends in race across U.S. cities. We show that the mean local information $J(X,Y)$ and the mutual information $I(X,Y)$ jointly supply simple and intuitive measures of spatial complexity for various cities. We also note an intriguing scaling relationship between spatial complexity and population density. We next evaluate our information-theoretic clustering method, and show that the neighborhoods it identifies can usefully shed light on traditional concerns of quantitative sociology, such as racial affinities and spatial concentration. Next, by defining a global evaluation measure on each clustering, we show that the mean local information $J(X,Y)$ does indeed measure the ``clusterability'' of a spatial compositional data set. 

	Section 4 is a discussion of our findings and prospects for future work. Finally, we include an Appendix supplying some mathematical details not present in the text, including a proof of our assertion that the mean local information is related to a fundamental statistical property of spatial compositional phenomena, and a thorough specification of algorithms and computations. 


