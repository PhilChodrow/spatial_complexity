\subsection{Neighborhoods, Segregation, and Information}

	The neighborhood in which an individual lives has profound impact on that individual's mental and physical well-being \cite{Ludwig2012, Ludwig2013}. Urban planners may aim to revitalize neighborhoods, and an entire cottage industry in sociology and epidemiology has grown around the idea of ``neighborhood effects'' (see, for example, \cite{DiezRoux2001, Sampson2002} for overviews of older work). Despite the persistent import of neighborhoods in social sciences, little consensus exists on how to define or demarcate neighborhoods in a non-arbitrary way. Many studies define ``neighborhoods'' by the boundaries supplied in their data sets, such as those determined by the U.S. Census \cite{Dietz2002}. Such an approach follows the limitations of the data provided, rather than a theoretically-principled concept of neighborhood. However, recent progress in non-arbitrary demarcations of urban form serve as proofs of concept for the theoretically-principled measurement of heterogeneity in urban form \cite{Rozenfeld2008,Rozenfeld2011}.

	We take up the question of neighborhoods in the context of segregation studies in sociology, and will argue that the measurement of segregation is intricately linked with the identification of spatial demographic structure. Many indices have been proposed in the last half-century to measure various aspects of segregation; see \cite{Massey1988, Reardon2002} for useful synthesizing overviews. Many such approaches have also taken Census-defined boundaries as constitutive of neighborhoods, but more recent work has focused ``spatialized'' measures that are independent of neighborhood demarcations\cite{Reardon2004, Roberto2015a, Roberto2015,Wong2004, Wong1999, Lee2008}. Other work has quantified the tendency of different social groups to cluster together \cite{Louf2015}. 

	The approach we develop is grounded in information measures of spatial demographic structure. Information theory maps the physical concept of structure to the epistemic concept of predictability. A complex system has enough structure to be at least partially predictable, but enough variability to make prediction challenging. Information-theoretic concepts have already been introduced in the study of cities, finding application in urban planning problems, in predicting population distributions, and in quantifying difference between neighborhoods \cite{Royal2014,Batty1974,Batty1976,Battya,Bettencourt2015,Webber1979}. Attractive features of information measures include their deep relationship to statistical inference \cite{Cover1991,Csiszzr2004}, their generalizability to a wide variety of phenomena, and their conceptual linkage of cities with the methods of statistical physics. In the context of the measurement of urban demographic structure, information measures provide answers to questions such as: 
	\begin{enumerate}
		\item Supposing you choose an individual randomly from a city, how accurately could I guess their race? 
		\item Supposing you then told me where that person lived within the city, how much would this new information increase my accuracy? 
	\end{enumerate}
	The following considerations connect these questions to the measurement of diversity and structure. If I can guess a random individual's race with 100\% accuracy given no information, then it must be the case that the city is monoracial; increased diversity reduces my ability to guess. If knowing where a person lives dramatically increased my ability to guess, then there must be substantial structural dependence of demographics on space, a prerequisite of segregation. 

	We make two primary contributions. The first is to the field of quantitative segregation studies in sociology. We formulate the \emph{mean local information} $J(X,Y)$ as a measure of spatiosocial structure. We then show that this novel measure, in concert with the mutual information $I(X,Y)$, comprise a unified, spatially-aware methodology for the measurement of diversity and segregation. These metrics possess many of the traditionally desirable properties of sociological indices, and can be easily computed from open data such as that provided by the U.S. Census. Our second contribution is to provide a simple, theoretically-motivated, and computationally-tractable method for agglomerating spatio-social data into ``natural neighborhoods'' across which demographic trends are approximately constant. This method has multiple applications. Sociologists can use this agglomeration procedure to study racially coherent neighborhoods, rather than the partially-arbitrary administrative boundaries such as those produced by the U.S. Census. Computational urban planners can use the method for \emph{dimension-reduction}, in which a large data set is made more computationally tractable while preserving spatio-social relationships of interest. Finally, urban physicists may be interested in the scaling behavior of the mutual information across different levels of aggregation in analogy with coarse-graining in statistical mechanics \cite{Bettencourt2015}. The hierarchical character of agglomerative clustering defines a non-arbitrary way to perform this aggregation. 

\subsection{Structure of the Essay}
	In Section 2, we motivate and develop the two core components of our mathematical framework. The first of these is the mean local information $J(X,Y)$, a measure of the intrinsic spatial complexity of a compositional phenomenon. In the context of racial trends in cities, the mean local information measures the granularity of racial neighborhood structure. It is low in a city like Detroit, which consists of a few large, monolithic tracts of constant racial composition. It is high in a city like Philadelphia, in which many (spatially) smaller neighborhoods are knit together in an intricate patchwork. The second component of our mathematical framework is a simple algorithm for identifying clusters that are both spatially and compositionally coherent through agglomerative hierarchical clustering and greedy information maximization. In the context of race in cities, this algorithm may be viewed as an automated means to identify ``natural'' neighborhoods. 

	In Section 3, we apply these techniques to the analysis of spatial trends in race across U.S. cities. We show that the mean local information $J(X,Y)$ and the mutual information $I(X,Y)$ jointly supply simple and intuitive measures of spatial complexity for various cities. We also note an intriguing scaling relationship between spatial complexity and population density. We next evaluate our information-theoretic clustering method, and show that the neighborhoods it identifies can usefully shed light on traditional concerns of quantitative sociology, such as racial affinities and spatial concentration. Next, by defining a global evaluation measure on each clustering, we show that the mean local information $J(X,Y)$ does indeed measure the ``clusterability'' of a spatial compositional data set. 

	Section 4 is a discussion of our findings and prospects for future work. Finally, we include an Appendix supplying some mathematical details not present in the text, including a proof of our assertion that the mean local information is related to a fundamental statistical property of spatial compositional phenomena, and a thorough specification of algorithms and computations. 
