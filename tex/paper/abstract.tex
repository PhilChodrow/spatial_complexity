
\abstract{We consider two related problems. First, the measurement of socio-spatial segregation is an ongoing concern of quantitative sociologists.  Second, the identification of natural, socio-economically defined neighborhoods arises in Census reporting, applied sociological analysis, and dimension reduction in urban computing. 

We unify these problems by developing a rigorous, information-theoretic approach to both, using open data on race in American cities as a case study. We formulate a suite of three information measures, including \emph{mean local information} $J(X,Y)$. The local information is a novel measure of complexity in unevenly organized cities, and we prove that it is closely related to the Fisher information of the underlying joint distribution. We show that the local information measures both spatial exposure as defined in segregation studies and the fineness of neighborhood structure. Unlike aspatial information measures, the mean local information clearly distinguishes between cities like Detroit--which is dominated by a few huge, monoracial superclusters--and cities like Philadelphia--which is an intricate patchwork of small, racially-distinct neighborhoods. Second, we provide a practical algorithm for identifying natural neighborhoods through greedy information maximization, and show that the fineness of neighborhood structure as identified by this algorithm is closely related to these information measures. This work provides a unified methodological foundation for further research into sociospatial phenomena, the dynamics of diversity over decades or days, and into scaling relations between urban density and neighborhood size.}

