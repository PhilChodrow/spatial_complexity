It is worthwhile to emphasize the characteristics of the data that make useful clustering possible. The unclustered data as provided by the U.S. Census presents a drawback: its boundaries are arbitrary. However, it also has an opportunity: its resolution is much higher than the resolution of the phenomena we aim to investigate. While Detroit contains over 1,800 blockgroups, it is clear that there are not 1,800 distinctive neighborhoods; rather, cluster analysis suggests that there are perhaps ten. We can think of our clustering method as a means of exchanging excess resolution for meaningful boundaries, with the mutual information quantifying how much meaning is lost in the exchange. 

We have formulated a novel measure of urban spatial complexity, and applied it to an analysis of spatial distributions of race in American cities. We emphasize that the mean local spatial variability $J(X,Y)$ is novel piece of a more complete information-theoretic characterization that also includes the entropy $H(Y)$ and the global variability $I(X,Y)$. Our results suggest two major directions of further exploration: 

One major physical question raised by our results is the origin of the scaling relation seen in \ref{fig:density}. This scaling relation may suggest a dynamical process of neighborhood formation and growth common to many American cities. If so, a physical understanding of this process in terms of individual-level behavior is called for. It is tempting to view the scaling relationship as an interplay of spatial preferential attachment and intrinsic limits to neighborhood density due to housing availability, but this view is speculative at this point. A theory to explain this behavior would be most welcome. 

Another question raised is more operational. In many urban planning contexts, the amount of available data may make direct computations prohibitive. In such contexts, it may be useful to construct a simplifying model of the data the data (such as a clustering) in order to reduce dimensionality and make ``the big picture'' more visible. When a particular demographic phenomenon (such as race) is under consideration, it may be important to ensure that the model preserves the major patterns of variation in the granular data. We conjecture that the local variability $J(X,Y)$ measures the ``model-ability'' of such data sets. Intuitively, a city like Detroit with relatively low $J(X,Y)$ and large, monoracial neighborhoods should be easily representable with relatively few modeled clusters, one for each of the relatively few major neighborhoods. On the other hand, relatively speaking, a city like Philadelphia with very intricate neighborhood structure may require substantially greater model complexity to represent without large loss of information. A promising course of further study is to design an information-theoretic clustering algorithm designed to model urban patterns, and then compare this algorithm's performance to $J(X,Y)$. 

A third question and final question relates to the time-dependence of spatial structure. On one time scale, daily movement around a city for work or leisure activities has the impact of ``mixing'' separated residents in public spaces, potentially leading to very different spatial patterns of racial difference. On another time scale, these measures may track how the spatial structure of cities evolves over decades, potentially shedding further light on the dynamics of city formation. 

A final question is....