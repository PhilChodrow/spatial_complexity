We have presented an information-theoretic framework for conceptualizing three core dimensions of urban diversity. Entropy measures global diversity; mutual information measures spatial evenness; and local information measures spatial exposure. Using information-based agglomerative clustering, we have also shown that these three measures are intimately related to the fineness of neighborhood structure. As \ref{fig:info_and_clusters} indicates, cities whose neighborhood structure can be easily characterized with just a few clusters tend to be those with either low diversity $H(Y)$ or low spatial exposure $J(X,Y)$, like Detroit. Cities with finer-grained spatial structure tend to have higher diversity $H(Y)$ and greater spatial exposure $J(X,Y)$, like Philadelphia. These tools are emminently practical; allowing both for the quantitative comparison of diversity profiles between cities (Figure \ref{fig:info_cross}) and more detailed analysis within a single city using naturally-defined neighborhoods (Figures \ref{fig:cluster_map} and \ref{fig:clusters}). We conclude by noting some areas for future work, and some intriguing system properties illuminated by our findings. 

While we have presented our information-theoretic framework in the context of racial residential segregation, these methods are substantially more general. Generalizing from race, information-theoretic methods applicable to any categorical variable or combinations thereof. It would be simple to conduct similar analysis for e.g. occupation type, or even to for a combination of race and occupation type. Information methods are somewhat less applicable in the case of continuous indicators \cite{Cover1991}. However, even in this case it would be possible to define interpretable qualitative categories such as \{``working class'', ``middle class'', ``upper class''\} and proceed with analysis. 

Generalizing from space, the remaining dimension of sociological interest is time. Many adults spend less than half of their waking hours at home \cite{employment_stats}, indicating that residential segregation is only a partial characterization of city-wide diversity. Fortunately, our ability to learn about daily patterns of human mobility is progressing at a rapid pace. Recent years have seen an enormous increase in the use of mobile devices, allowing the passive collection of digital traces. These traces can be processed, analyzed, and validated to derive insight into daily activity patterns \cite{Widhalm2015,Yang,Jiang2013,Jiang2012c}. In the context of these developments, the information-theoretic suite of tools is attractive in its generality. Given appropriate data, only minimal mathematical changes are necessary: 
\begin{itemize} 
	\item The entropy $H(Y)$ remains unchanged. 
	\item Spatiotemporal evenness is the degree to which spatiosocial distributions change throughout the day. It can be measured by the global spatiotemporal mutual information $I([T,X];Y)$ between time $T$, space $X$, and demographics $Y$. Also of interest is the quantity $I(X,Y|T=t)$, which measures spatial unevenness at a given time of day $t$. 
	% \item The local information $J(T,X,Y)$ retains its primary definition, but now splits into two parts:  $J(T,X,Y) = J(T,Y) + J(X,Y)$. The first term on the right measures how rapidly racial trends at fixed locations change with time, while the second term is the familiar (spatial) local information. 
	% Check this: not sure it's actually true. 
\end{itemize}
It is also possible to define a greedily information-maximizing spatiotemporal clustering algorithm. Such an algorithm may be of special interest in temporal analysis on longer time-scales. For example, given many years of Census data, spatiotemporal clustering would allow the identification of neighborhoods and a characterization of their demographic and spatial evolutions. 

The scaling relationship between the local information $J(X,Y)$ and the population density as shown in Figure \ref{fig:density} also deserves further investigation. Keeping in mind that $J(X,Y)$ measures the fineness of neighborhood structure, the upshot of Figure \ref{fig:density} is that, controlling for evenness, denser cities tend to have more racially-distinct neighborhoods per area. While further analysis is necessary to make this observation precise, it is intriguing to postulate a dynamical process of racial neighborhood formation and growth common to many American cities. It may be, for example, that an explanation is possible in terms of the interplay between spatial preferential attachment and infrastructure constraints. 